\chapter{Discussion}
In this thesis, we proposed a novel approach to solving the game of Briscola by using the PPO algorithm with function approximation to learn the optimal policy. To evaluate the performance of our latest agent, BriscolaBot-v3, we developed a website, accessible at \url{https://replit.com/@LorenzoCavuoti/BriscolaBot}, where the agent could play against human players.\\\\
Our results showed that BriscolaBot-v3 achieved a 59\% win rate, with a 90\% confidence interval of [0.53, 0.65], against average human players, while against self-declared expert players BriscolaBot-v3 has won 52\% of the games, with only one expert player, being able to beat it 12-11; these results confirm the effectiveness of our method.\\\\
The training of our agent was performed on a standard 4-core computer, ensuring reproducibility of our results and making it feasible to train a competitive agent on any machine in a reasonable amount of time.

\section{Future Work}
We have identified several potential improvements to further enhance the performance of the agent, including:
\begin{itemize}
    \item \textbf{Exact solving of endgame positions:} when all cards have been extracted the cards of the opponent are known, so it becomes possible to solve the game of Briscola exactly with minimax search. This improvement can be achieved at a low computational cost, since the number of possible ways of playing out a given endgame is only 36, which can be further reduced through the application of dynamic programming techniques.
    \item \textbf{Improve agent performance with MCTS:} previous methods in the game of Briscola used Monte Carlo Tree Search (MCTS) to identify the best move \cite{Briscola-mcts-Playing-Algorithm, villa2013-briscola-mcts}. However, these approaches relied on a model of the opponent rather than learning from it, such as assuming the opponent throws a random card from the set of unseen cards. We suggest improving this approach by training a model that predicts the opponent's possible move in a given situation, which can then be used to perform MCTS search.
    \item \textbf{Approximating the Nash equilibrium:} we can use imperfect information game-solving approaches to directly find or approximate the Nash equilibrium of the game of Briscola. These approaches have been used successfully in poker to beat the world's best players \cite{libratus}.
    \item \textbf{Extend the observation space of the agent:} currently, the agent only observes the cards that have been played, but not the order in which they were played. By observing the order in which cards were played, the agent could potentially better predict the cards that the opponent has in their hand. For example, if the agent plays an ace of cups and the opponent does not use briscola to take the trick, it is likely that the opponent does not have any briscola in their hand. This information could allow the agent to be more aggressive in the following tricks.
\end{itemize}
Possible other directions for future work include applying the same technique to other Italian card games, such as Scopa, Tressette, Macchiavelli, and Scala 40. Additionally, the approach can be extended to Briscola when played with multiple players, as the game can be played by up to six players in teams of two or three.
